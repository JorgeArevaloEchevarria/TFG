\chapter{Conclusiones y Trabajo Futuro}
\label{cap:conclusiones}

Una vez finalizado el proyecto, podemos decir que el objetivo de crear una aplicación para el mantenimiento del \textit{stock} de restaurantes se ha cumplido. Se ha quedado una interfaz muy intuitiva y sencilla de utilizar, que era el objetivo principal. Los objetivos alcanzados han sido:

\begin{itemize} 

\item Creación de distintas vistas para gestionar el \textit{stock} de un restaurante desde la visión del propietario del local.

\item Diseño de una base de datos para almacenar los ingredientes, bebidas y platos donde poder acceder y seleccionar la información.

\item Implementación de la aplicación móvil para que se puedan tomar nota desde cada mesa y simultáneamente se actualice el \textit{stock} del restaurante.

\item Implementación de la aplicación web para poder gestionar el \textit{stock} de los ingredientes, bebidas y platos del restaurante y se actualice la base de datos al momento.
 
\item Diseñar un aspecto visual agradable que haga una mejor experiencia del usuario, tanto en la aplicación web como en la aplicación móvil.

 \end{itemize}

Todas las funcionalidades se han desarrollado de manera independiente, tratando de tener un código limpio y organizado, lo que nos permitirá añadir nuevas funcionalidades de manera cómoda y sencilla.

En cuanto a la funcionalidad de las aplicaciones, se han solventado los requisitos mínimos para poder gestionar un restaurante, pero es fácil añadir nuevas características que sean requeridas en cada establecimiento en particular.

Como trabajo futuro nos gustaría enumerar una serie de ideas que se podrían implementar:
\begin{itemize}

 \item \textbf{Limitaciones de Firebase}. Se ha usado Firebase para la implementación, ya que es una base de datos muy práctica para proyectos pequeños. 
Sin embargo, se ha usado un plan de datos gratuito, el cual tiene límite de consultas y de usuarios simultáneos. Para implementarlo en un restaurante se debería pasar a un plan de pago. Si se quiere usar para grandes cadenas de restaurantes, por ejemplo, habría que estudiar el rendimiento y la posibilidad de cambiar a otra base de datos

\item \textbf{Mejoras en el dominio}. Al igual que en la base de datos, se ha usado una página web que nos ofrece un dominio por tiempo limitado. Habría que estudiar la posibilidad de comprar y hacernos con el dominio \textit{TakeOrderAdmin.es.}

\item \textbf{Alertas.}
Se podría añadir una funcionalidad para notificar mediante un email al encargado del \textit{stock} de las alertas asociadas a los ingredientes y bebidas.

\item \textbf{Precio y proveedores.}
Implementar campos para guardar los precios y los proveedores de los ingredientes y bebidas sería muy útil.

\item \textbf{Menús del día.}
Sería interesante tener una serie de menús del día ya predefinidos, por ejemplo para cada día de la semana, y poder cargarlos rápidamente.

También se podría añadir una funcionalidad para poder acceder a estos menús del día o a toda la carta mediante un código QR o poder imprimirlos con un formato adecuado.

\item \textbf{Creación de usuarios}.
Se podrían integrar la creación usuarios, de tal forma que podamos restringir algunas funcionalidades a ciertos trabajadores, o poder mostrar alguna información sensible como precios y proveedores solo al administrador del \textit{stock}.
\end{itemize}

