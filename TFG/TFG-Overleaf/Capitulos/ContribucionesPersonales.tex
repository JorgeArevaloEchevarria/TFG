\chapter*{Contribuciones Personales}
\label{cap:contribucionesPersonales}
\addcontentsline{toc}{chapter}{Contribuciones Personales}

Desde un inicio decidimos repartirnos el trabajo de manera que los dos tuviéramos la misma carga. Nos hemos compenetrado muy bien fortaleciendo los puntos fuertes de cada uno. Aunque se ha dividido el trabajo de desarrollo de las aplicaciones, hemos estado siempre en contacto, consultándonos dudas y proponiendo nuevas ideas el uno al otro.


\section*{Jesús Martín Moraleda}
Al principio del proyecto, consideramos invertir un tiempo en estudiar todas las tecnologías que podíamos utilizar, ya que el desarrollo web y móvil era prácticamente nuevo para nosotros. Me encargué de probar e informarme de ventajas e inconvenientes de cada una y decidir las que íbamos a usar.
Del mismo modo investigué que base de datos nos convenía usar y como conectar nuestras aplicaciones con esta.

Una vez teníamos claras las tecnologías, me dediqué a crear el proyecto de la base de datos en \textit{Firebase} y, junto a Jorge, a conectar las aplicaciones con la base de datos. Una vez las dos aplicaciones se comunicaban correctamente con la base de datos, me encargué de crear un repositorio de GitHub para cada proyecto. Una vez hice los primeros \textit{commits} con la estructura del proyecto y la conexión con la base de datos, Jorge empezó con el desarrollo de la aplicación móvil y yo me dediqué al desarrollo de la web.

Con todo esto listo, empecé con el desarrollo de la página web con la ayuda de los \textit{mockups} que había realizado mi compañero. Empecé desarrollando la pantalla de los ingredientes. Mi objetivo era ir desarrollando las vistas completas y que funcionaran correctamente. Luego seguí con las pantallas de bebidas e ingredientes, y mas tarde realice las vistas del menú del día y las alertas.

Mientras hacíamos todo esto, teníamos reuniones con Mercedes, que nos fue guiando en el desarrollo del proyecto. Nos dió libertad para usar las tecnologías que quisieramos, y las funcionalidades de la aplicación las debatíamos entre los tres en estas reuniones.

En mitad del desarrollo, me puse a investigar como podíamos desplegar la página web en un dominio, para que se pudiera acceder de otra manera que desde nuestros ordenadores. Esto me llevó un tiempo ya que no habíamos tenido que desplegar una página web anteriormente.

En esta fase de desarrollo, Jorge y yo debatíamos juntos las dudas y las cosas nuevas que queríamos implementar en las dos aplicaciones. Veíamos como poder implementarlas y que tuvieran concordancia las funcionalidades de una aplicación con la otra.

La memoria la fuimos completando a medida que íbamos avanzando en las fases. Mercedes nos fue ayudando a como estructurarla e implementarla

\section*{Jorge Arévalo Echevarría}

Durante la fase de análisis realizábamos reuniones con nuestra tutora, para ir dando forma y moldeando en que dirección queríamos llevar el proyecto, definiendo los requisitos y las funcionalidades que queríamos llevar a cabo. Nuestra tutora nos dio completa libertad para la selección de entornos de desarrollo o lenguajes de programación, dejándonos escoger con los que nos sintiésemos más cómodos. Además, hicimos un diseño previo de como sería nuestra base de datos que se encargaría de almacenar la información de nuestras aplicaciones. Así que por último empezamos a planificar y dividir el trabajo para investigar que entornos utilizar para el proyecto.

Durante la fase de investigación mi labor consistió en el estudio y diseño de la base de datos y de la aplicación móvil.

Respecto a la base de datos, empecé investigando los entorno conocidos y utilizados en las distintas asignaturas de la carrera como "Bases de Datos" en la que utilizábamos el entorno Oracle Database. Después vimos que era muy buen entorno para desarrollar la base de datos a nuestro gusto, pero nos surgió el problema de que los ejemplos que estábamos acostumbrados a hacer en clase trabajábamos en local, y nosotros necesitábamos una base de datos remota para poder acceder a los datos desde la página web. Jesús me informo sobre "Firestone Firebase" siendo muy fácil e intuitivo para proyectos a pequeña escala, teniendo un gran resultado en el desarrollo de sus proyectos, además de ser un entorno gratuito. Así, Jesús se encargó del diseño e instalación de la base de datos para el desarrollo de nuestro proyecto.

Respecto a la aplicación móvil, yo ya tenía experiencia respecto a entornos de desarrollo, ya que utilice Android Studio en la asignatura optativa de "Programación de aplicaciones para dispositivos móviles". Lo que más me gusto fue la libertad que proporciona a la hora de diseñar las vistas, además de usar el lenguaje de programación Java, con el que todos estamos familiarizados. Respecto a las vistas para la aplicación móvil, decidimos realizar unos \textit{mockups} para ir diseñando y plasmando las ideas preconcebidas que teníamos en la cabeza, de como queríamos que fuese la aplicación y que funcionalidades queríamos que llevase a cabo.

En la fase de configuración y de instalación, me encargué de la instalación de la última versión de Android Studio en nuestros dispositivos. También generamos un repositorio en Github para comprobar las actualizaciones realizadas, además de controlar y guardar las versiones anteriores de nuestro proyecto de la aplicación móvil.

La fase de desarrollo se puede decir que fue la más compleja de las fases, ya que fue la que nos llevó más tiempo y esfuerzo. Durante esta fase me encargué de llevar a cabo la implementación de las ideas que habíamos preconcebido para la aplicación móvil. Lo primero que hice fue enlazar mi proyecto al repositorio de Github, para poder controlar la subida y actualizaciones del proyecto, o en caso de cometer un error, poder recuperar una versión que funcionase de forma correcta.
Después, uno de los pasos más importantes fue el de realizar la conexión del proyecto con la base de datos Firestore database, para poder obtener la información o editar campos dentro de la base de datos. A partir de aquí desarrollé todas las vistas y funcionalidades de la aplicación móvil, en colaboración con Jesús, mientras desarrollaba la página web, para ir dando forma al producto final que queríamos obtener.

La memoria del proyecto la fuimos completando mientras íbamos avanzando en las distintas fases del proyecto. Gracias a la corrección y supervisión de nuestra tutora que nos daba indicaciones de qué partes debíamos modificar o mejorar, encarrilando y dirigiendo la memoria para conseguir un nivel apto de entrega. 



