\chapter*{Conclusions and Future Work}
\label{cap:conclusions}
\addcontentsline{toc}{chapter}{Conclusions and Future Work}

Once the project is finished, we can say that the objective of creating an application for the maintenance of restaurants has been achieved. It has resulted in a very intuitive and easy to use interface, which was the main objective. Some of the objectives have been achieved:

\begin{itemize} 

\item Creation of different views to manage the stock of a restaurant from the owner's point of view.
 
\item Design of a database to store the ingredients, drinks and plates where the information can be accessed and selected.

\item Implementation of the mobile app so that waiters can take notes from each table and simultaneously update the restaurant's stock.

\item Design a friendly visual appearance that makes a better user experience, both in the web application and in the mobile application.

 \end{itemize}

All the functionalities have been developed independently, trying to have a clean and organized code, which will allow us to add new functionalities in a comfortable and simple way.

As for the functionality of the applications, the minimum requirements to manage a restaurant have been solved, but it is easy to add new features that are required in each particular establishment.

As future work we would like to list a number of ideas that could be implemented:
\begin{itemize}

\item \textbf{Firebase Limitations}
Firebase has been used for the implementation as it is a very practical database for small projects. 
First of all a free data plan has been used, which has a limit of queries and simultaneous users, to implement it in a restaurant it would be necessary to change to a paid plan, if you want to use it for large restaurant chains, for example, you would have to study the performance and the possibility of changing to another database.

 \item \textbf{Domain improvements}
As in the database, we have used a web domain that offers us a domain for a limited time. It would be necessary to study the possibility of buying and getting the domain TakeOrderAdmin.es 

 \item \textbf{Alerts}
It would be possible to add a funcionality to notify by email to he owner of the restaurant when there is some ingredient or drink in alert.

 \item \textbf{Price and providers}
Implementing fields to store prices and providers of ingredients and drinks would be very useful.

 \item \textbf{Menus of the day}
It would be interesting to have a predefined set of menus of the day for example for each day of the week, and to be able to load them quickly.

You could also add a functionality to access these menus of the day or the entire menu via a QR code or to print them in a suitable format.

\item \textbf{User creation}
The creation of users could be created, so that we can restrict some functionalities to certain workers, or be able to show some sensitive information like prices and suppliers only to the restaurant owner.  \ref{cap:conclusiones}.

\end{itemize}

